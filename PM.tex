\documentclass{VUMIFPSkursinis}
\usepackage{algorithmicx}
\usepackage{algorithm}
\usepackage{algpseudocode}
\usepackage{amsfonts}
\usepackage{amsmath}
\usepackage{bm}
\usepackage{caption}
\usepackage{color}
\usepackage{float}
\usepackage{graphicx}
\usepackage{listings}
\usepackage{subfig}
\usepackage{array}
\usepackage{wrapfig}
\usepackage{tabu}
\usepackage{longtable}
\usepackage{eurosym}

\newcolumntype{L}[1]{>{\raggedright\let\newline\\\arraybackslash\hspace{0pt}}m{#1}}
\newcolumntype{C}[1]{>{\centering\let\newline\\\arraybackslash\hspace{0pt}}m{#1}}
\newcolumntype{R}[1]{>{\raggedleft\let\newline\\\arraybackslash\hspace{0pt}}m{#1}}


% Titulinio aprašas
\university{Vilniaus universitetas}
\faculty{Matematikos ir informatikos fakultetas}
\department{}
\papertype{Projektų valdymo laboratorinis darbas}
\title{Projektų valdymas}
\titleineng{Project management}
\status{1 kurso magistratūros studentai}
\author{Šarūnas Kazinieras Buteikis}
\secondauthor{Matas Savickis}
\thirdauthor{Irmantas Ivanauskas}
\supervisor{dr. Giedrius Slivinskas}
\date{Vilnius – \the\year}

% Nustatymai
% \setmainfont{Palemonas} % Pakeisti teksto šriftą į Palemonas (turi būti įdiegtas sistemoje)
\bibliography{bibliografija}

\begin{document}

\maketitle

\tableofcontents

\section{Project Objectives and Description}
	\subsection{Project Description}
		Project is called ,,Real estate rental self-service". 
		Our self-service system enables landlords and tenants to create a rent contract and manage it.

	\subsection{Project Objectives}
		\begin{enumerate}
			\item{In 9 months from the start of the project, launch a self-service website.}
			\item{By the begining of the project development, assemble software development team consisting of two mid-senior level developers and one QA tester.}
			\item{In 6 months from the self-service system launch accumulate 15000 active contracts registered in the app.}
			\item{In 12 months from the self-service system launch accumulate 30000 active contracts registered in the app.}
		\end{enumerate}

	\subsection{Expected benefits}
		\begin{enumerate}
			\item{To make a renting process more transperate and convinient.}
			\item{Create self-sustaining business model with contract and transaction fees.}
		\end{enumerate}

	\subsection{Stakeholders}
		\begin{enumerate}
			\item{Investors}
			\item{Employees}
			\item{Tenants}
			\item{Landlords}
		\end{enumerate}


\section{Scope Statement}
	\subsection{Project scope Description}
		\begin{enumerate}
			\item{Real estate rental self-service website will be built and depolyed with following features.}
				\begin{enumerate}
					\item{Ability to register and login into self-service account.}
					\item{Ability to create rent contract between two users.}
					\item{Ability to pay rent fee via electronic bank.}
					\item{Ability to auto-pay for rent via electronic bank.}
					\item{Ability to rate landlords, tenants, real estate.}
					\item{Ability to comment about landlords, tenants, real estate.}
				\end{enumerate}
		\end{enumerate}

	\subsection{Project Deliverables}
		\begin{enumerate}
			\item{Deployed and ready-to-use rental contract self-service website.}
			\item{Product source code.}
			\item{User manual.}
			\item{Support manual.}
		\end{enumerate}

	\subsection{Project Acceptance Criteria}

	\subsection{Project Scope Exclusion}
		\begin{enumerate}
			\item{Self-service system does not solve legal problems between landlord and tenant.}
			\item{Self-service system does not work as rent adveritisment site.}
			\item{Self-service system only supports Lithuanian electronic bank services as payment method.}
			\item{Self-service system only works with Lithuanian landlords and Lithuanian realestate.}
		\end{enumerate}

	\subsection{Project Constraints and Assumptions}

\section{Stages and Schedule}
	\subsection{Project initiation}
	\textbf{Deadline 2020-10-01}
	\subsection{Project planning}
	\textbf{Deadline 2020-12-31}
	\subsection{Project execution}
	\textbf{Deadline 2021-05-01}
	\subsection{Project performance} 
	\textbf{Deadline 2021-06-01}
	\subsection{Project closure}
	\textbf{Deadline 2021-07-01}
	Team at this point is terminated. Once a project is complete, Project manager organize meeting to evaluate what went well in a project and identify failures.

\section{Project Team}
	\subsection{Roles}
	\begin{enumerate}
		\item{Project manager} - develops project plan, lead and manage project team
		\item{Scrum master} - ensures that in software development process are used best agile practices, manages the product backlog, helps with communication
		\item{Business analyst} - responsible for ensuring that requirements are captured and documented correctly before a solution is developed.
		\item{Software architect} - collaborates with analyst to determine software requirements, creates software design and resolves issues according to it.
		\item{Senior software developer} - leads the development process, ensures that development team is providing clean and efficient code
		\item{Software developer} - develops system by writing clean and efficient code
		\item{Tester} - identifies target test items, defines and executes test cases, gathers and manages test data, ensures, that the solution meets business requirements and it is free of errors and defects.
	\end{enumerate}
	
	\subsection{Team size}
	Team is formed by 1 project manager, 1 scrum master, 1 business analyst, 1 software architect, 1 senior software developer, 3 software developers and 2 testers.

\section{Budget}
This section includes all necessary costs for the project: employee salaries, equipment, licenses, third-party services, etc.

\begin{center}
	\setstretch{1.0}
	\small
	\begin{longtable}{|L{2cm}|C{1cm}|C{1cm}|C{1cm}|C{1cm}|C{1cm}|C{1cm}|L{1cm}|}
		\caption{Budget item and cost across the duration of the project}
		\label{table:EmployeeSalary}
		\\ \hline
		Budget item &
		\multicolumn{1}{c|}{Month 1} &
		\multicolumn{1}{c|}{Month 2} &
		\multicolumn{1}{c|}{Month 3} &
		\multicolumn{1}{c|}{Month 4} &
		\multicolumn{1}{c|}{Month 5} &
		\multicolumn{1}{c|}{Month 6} &												
		\multicolumn{1}{c|}{\textbf{Total, \euro}} \\ \hline
		Project Manager Salary &
		3000&
		&
		3000&
		3000&
		3000&
		3000&
		\textbf{18000}\\ \hline
		Scrum Master Salary &
		&
		&
		&
		&
		&
		&
		\textbf{xxxx}\\ \hline
		Business Analyst Salary &
		&
		&
		&
		&
		&
		&
		\textbf{gggg}\\ \hline
		Software architect Salary &
		&
		&
		&
		&
		&
		&
		\textbf{18000}\\ \hline
		Senior software developer Salary &
		&
		&
		&
		&
		&
		&
		\textbf{18000}\\ \hline
		Software developer Salary &
		&
		&
		&
		&
		&
		&
		\textbf{ggg}\\ \hline
		Test Analyst Salary &
		1800&
		1800&
		1800&
		1800&
		1800&
		1800&
		\textbf{ggg}\\ \hline										
\end{longtable}
\end{center}



\section{Risks and Responses}
Risks are divided into two main categories: positive risks and negative risks. All risks have a probability and an impact to the project.

\subsection{Negative Risks}
Table \ref{table:NegativeRisksReponses} covers risks with negative impact to the project and the recommended response/s to mitigate said risk.

\begin{center}
	\setstretch{1.0}
	\small
	\begin{longtable}{|C{1cm}|L{4.55cm}|C{2cm}|C{2cm}|L{4.55cm}|}
		\caption{Negative Risks and Responses}
		\label{table:NegativeRisksReponses}
		\\ \hline
		Risk code &
		\multicolumn{1}{c|}{Risk Description} &		
		Probability &
		Impact &
		\multicolumn{1}{c|}{Response} \\ \hline
		R1 &
		\textbf{Unfamiliarity with development process used}. Team members unfamiliar with development process will have a slower workflow resulting in longer delivery times&
		Low &
		Medium &
		\textbf{Team will perform a bi-weekly retrospective meeting} at the end of the week. These retrospectives should create a safe space for team members to share their honest feedback on what's going well and what isn't, and to generate a discussion around what could be improved next time regarding the development process during the sprint.		\\ \hline
		R2 &
		\textbf{Inefficiency of the development process}. inefficient development process lead time and lowers deliverable quality.&
		Low&
		High&
		{\textbf{Perform bi-monthly squad health checks} to gather what each team member feels about the following aspects: delivering value, easy to release, health of codebase, mission, speed, suitable process, support and teamwork. Results will allow to make necessary plans on which aspect/s to improve resulting in higher team motivation and synergy as well as adjusting the unfamiliar.\newline \textbf{Collect data of KPIs}, such as, registered defects per day, commits per day, what tasks are assigned to which member, lead time, etc. in order to follow the current state of the development process and see if made changes affected those KPIs}\\ \hline		
		R3 &
		\textbf{Low team motivation, synergy and/or coordination}. Low motivation, synergy and/or coordination results in slower lead time, lack of innovation and is prone to higher no. of errors during the execution of the project &
		Medium&
		Medium&
{\textbf{Perform bi-monthly squad health checks} to gather what each team member feels about the following aspects: fun, learning, pawns or players, support and teamwork. Results will allow to make necessary plans on which aspect/s to improve resulting in higher team motivation and synergy.\newline \textbf{Promote an environment to nurture T-shaped professionals} resulting in increased team synergy and coordination}\\ \hline
		R4 &
		\textbf{Insufficient team skill/knowledge regarding tech-stack}. A Lack of skill and knowledge of the tech stack used in a project results in higher amount and severity of bugs, slower development time and bad practices&
		Medium&
		High&
		Perform a different tech analysis in order to gauge the difficulty to master said tech, it's ease of use, functionality, how comfortable is the team to use said tech and so on. Performed analysis will let to pick the tech that is not only suited, but is also familliar to the team members.
		Organise knowledge sharings/workshops at the begining of the project in order for the team to fammiliarize themselves to the tech-stack\\ \hline					
	\end{longtable}
\end{center}

\subsection{Positive Risks}
Table section risks with positive impact to the project and the recommended response/s to achieve those risk.
\end{document}
