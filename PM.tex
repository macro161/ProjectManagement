\documentclass{VUMIFPSkursinis}
\usepackage{algorithmicx}
\usepackage{algorithm}
\usepackage{algpseudocode}
\usepackage{amsfonts}
\usepackage{amsmath}
\usepackage{bm}
\usepackage{caption}
\usepackage{color}
\usepackage{float}
\usepackage{graphicx}
\usepackage{listings}
\usepackage{subfig}
\usepackage{array}
\usepackage{wrapfig}
\usepackage{longtable}
\usepackage{eurosym}


\newcolumntype{L}[1]{>{\raggedright\let\newline\\\arraybackslash\hspace{0pt}}m{#1}}
\newcolumntype{C}[1]{>{\centering\let\newline\\\arraybackslash\hspace{0pt}}m{#1}}
\newcolumntype{R}[1]{>{\raggedleft\let\newline\\\arraybackslash\hspace{0pt}}m{#1}}

% Titulinio aprašas
\university{Vilniaus universitetas}
\faculty{Matematikos ir informatikos fakultetas}
\department{}
\papertype{Projektų valdymo laboratorinis darbas}
\title{Projektų valdymas}
\titleineng{Project management}
\status{1 kurso magistratūros studentai}
\author{Šarūnas Kazimieras Buteikis}
\secondauthor{Matas Savickis}
\thirdauthor{Irmantas Ivanauskas}
\supervisor{dr. Giedrius Slivinskas}
\date{Vilnius – \the\year}

% Nustatymai
% \setmainfont{Palemonas} % Pakeisti teksto šriftą į Palemonas (turi būti įdiegtas sistemoje)
\bibliography{bibliografija}

\begin{document}

\maketitle

\tableofcontents

\section{Project Objectives and Description}
	\subsection{Project Description}
		Project is called ,,Real estate rental self-service". 
		Our self-service system enables landlords and tenants to create a rent contract and manage it to ensure transparent and trackable communication between parties.

	\subsection{Project Objectives}
		\begin{enumerate}
			\item{In 6 months from the start of the project, launch a self-service website.}
			\item{By the begining of the project development, assemble software development team consisting of two mid-senior level developers and one QA tester.}
			\item{Launch a marketing campaigns one month before project launch date.}
		\end{enumerate}

	\subsection{Expected benefits}
		\begin{enumerate}
			\item{Ability to track your rent payments, integrate those payments with other bulk payment systems like ,,Viena sąskaita" and ,,Mano gile".}
			\item{To make rent deposits tracked more easily and evidential help to solve a dispute when landlord doesn't want to give back deposit money after end of contract.}
			\item{Create easy way to track real estate condition before and after rent contract to avoid damaged property disputes by taking a photographs of a real estate.}
			\item{Enable tenant to photograph utility meter and send it to landlord so landlord doesn't have to visit apartment each month.}
			\item{Create self-sustaining business model with contract and transaction fees.}
		\end{enumerate}

	\subsection{Benefit calculations}

	\subsection{Stakeholders}
		\begin{enumerate}
			\item{Investors - people who will invest money into self-service project to make it happen. 
				We will conduct monthly meetings with investors to present project progress and future projections.
				Feedback will be collected after each monthly meeting and actions will be taken to address investor's concerns.}
			\item{Employees - people who will work on self-service project.
				Each employee gets relevant information about project status as soon as it becomes available.
				Employee get monthly salary and if investors doesn't leave the project the employee's isn't fired.}
			\item{Tenants - people who will rent real estate from landlords.
					Uses application services to track payments, deposits, takes pictures of utility counters.
					Tenants gets monthly newsletter to the email about new feature in the system or any related information.}
			\item{Landlords - people who will lease real estate to the tenants.
					Track payments, deposits, see utility counter pictures and pays real estate utility bills via other systems such as ,,Viena sąskaita".}
		\end{enumerate}

	\subsection{Progress tracking}
		To track the progress of a system we create a application feature milestone schedule.
		During the investors meeting we present milestones we where able to achieve and adjustments made to the schedule and why those changes needed to be made.
		We will track one milestone per Agile sprint.
		Employees will be informed about milestone the same time a investors.


\section{Scope Statement}
	\subsection{Project scope Description}
		\begin{enumerate}
			\item{Real estate rental self-service website will be built and deployed with following features.}
				\begin{enumerate}
					\item{Ability to register and login into self-service account.}
					\item{Ability to create rent contract between two users.}
					\item{Ability to update contract details if both parties agree.}
					\item{Ability to pay rent fee via electronic bank.}
					\item{Ability to auto-pay for rent via electronic bank.}
					\item{Ability to rate landlords, tenants, real estate.}
					\item{Ability to comment about landlords, tenants, real estate.}
					\item{Ability to track payment details.}
					\item{Ability for user to see what data is tracked about him.}
					\item{Ability for user to delete all the data about him to comply with GDPR (liet. BDAR)}
				\end{enumerate}
			\item{Our company will conduct a marketing campaigns via internet advertisement. Advertisement will begin one month before product launch.}
		\end{enumerate}

	\subsection{Project Deliverables}
		\begin{enumerate}
			\item{Deployed and ready-to-use rental contract self-service website.}
			\item{Product source code.}
			\item{User manual.}
			\item{Support manual.}
			\item{Prepared strategy for system's end-of-life.}
			\item{QA created tests cases.}
			\item{Automated tests source code.}
			\item{Test plan.}
			\item{UX design elements.}
		\end{enumerate}

	\subsection{Project Acceptance Criteria}
		\begin{enumerate}
			\item{All the functional and non-functional requirements should be implemented unless said otherwise on the agreement between development team and investors.}
			\item{Test cases defined as high priority in the test must pass with a zero tolerance fail rate; defined as medium priority - 5\% fail rate; defined as low - 10\% fail rate.}
			\item{System UI design must adhere to UX defined UI layout, colour palette, font prototype, etc.}
			\item{System will be constrained to web app with responsive design.}
		\end{enumerate}


	\subsection{Project Scope Exclusion}
		\begin{enumerate}
			\item{Self-service system does not solve legal problems between landlord and tenant.}
			\item{Self-service system does not work as rent advertisement site.}
			\item{Self-service system only supports Lithuanian electronic bank services as payment method.}
			\item{Self-service system only works with Lithuanian landlords and Lithuanian real estate.}
			\item{Self-service system doesn't support Internet Explorer browser explicitly.}
			\item{All rented real estate must be in Lithuania.}
			\item{Self-service system does not solve conflicts between tenant and landlord.}
			\item{Self-service system supported languages are lithuanian, english and russian.}
		\end{enumerate}

	\subsection{Project Constraints}
		\begin{enumerate}
			\item{Self-service system, creators and owners are constrained by Lithuanian and European union laws.}
			\item{Project is bounded by 200k euro budget.}
			\item{By the 2021-07-01 project should be finished, launched into the production and project team terminated.}
		\end{enumerate}

		

\section{Project Stages}

	\subsection{Project initiation}
	\textbf{Deadline 2020-11-01}
	Within the initiation phase, the business problem or opportunity is identified, a solution is defined, a project is formed, and a project team is appointed to build and deliver the solution to the customer. A business case is created to define the problem or opportunity in detail and identify a preferred solution for implementation. 
	\subsection{Project planning}
	\textbf{Deadline 2020-12-31}
	The planning phase is when the project plans are documented, the project deliverables and requirements are defined, and the project schedule is created. It involves creating a set of plans to help guide your team through the implementation and closure phases of the project. The plans created during this phase will help us to manage time, cost, quality, changes, risk, and related issues.
	\subsection{Project execution}
	\textbf{Deadline 2021-06-01}
	Is the implementation processes that is the act of doing or performing the works and activities in accordance with agreed plans and procedures to satisfy the specifications and contractual requirements. The Project Execution is the performing the project scope of works and activities in accordance with the project baselines, plans and procedures with the resource, interface, change, schedule, cost, risk, quality, safety and environment management, and other contractual requirements.
	\subsection{Project performance} 
	\textbf{Deadline 2021-06-01}
	Occurring at the same time as the execution phase, this one mostly deals with measuring the project performance and progression in accordance to the project plan. Scope verification and control occur to check and monitor for scope creep, and change of control to track and manage changes to project requirement. Calculating key performance indicators for cost and time are done to measure the degree of variation, if any, and in which case corrective measures are determined and suggested to keep a project on track.
	\subsection{Project closure}
	\textbf{Deadline 2021-07-01}
	Team at this point is terminated. Once a project is complete, Project manager organize meeting to evaluate what went well in a project and identify failures.
	\subsection{Project maintenance}
	\textbf{After closure}
	For the further project ownership, development and maintaining there will be created smaller team of project manager, tester and developer.

\section{Project Execution Schedule}

	\subsection{Tasks}
	\begin{enumerate}
		\item{Create user manual} - prepare documentation for users to help them to use the system
		\item{Prepare development environments} - prepare environments for development and production deployment
		\item{Create UI wireframes} - create key concepts of UI
		\item{Make legal analysis} - make analysis of rent laws and legal environment in Lithuania
		\item{Development of user authorization} - create ability to login and registration to our system
		\item{Development of database structure} - choose DBMS and make core database structure
		\item{Integrations with Banks} - integrate ability to check persons identity with e-banking systems
		\item{Ability to track payment details} - create ability for user to track his payment details
		\item{Chat or messaging} - ability to write a message for tenant/landlord
		\item{Comment section about tenant/landlord/real estate} - ability to write and read comments about tenant/landlord/real estate
		\item{Rate tenant/landlord/real estate} - ability to rate and see rating of tenant/landlord/real estate
		\item{Ability to create rent contract} - users should be able to initiate new rent contract from application then sign it.
		\item{Ability to update or terminate rent contract} - users should be able to change terms and conditions of rent contract if both - tenant and landlord - accept these amendments
		\item{Administration tool to manage users} - there should be a tool for administrator to view user data, change their information, block user
		\item{Test execution} - test case execution
		\item{Ability to delete all user data} - there should be ability to delete all user data because of GDPR
		\item{Ability to see for user, what data about him is tracked} - there should be ability to see all data that system owns about user because of GDPR
		\item{Clarify requirements} - clarify and document requirements for the system
		\item{Test cases preparation} - create test cases for 
		\item{Integration with social networks} - ability to login or register to system using social networks (Facebook, Google)
		\item{Integration with Smart-ID} - ability to identify user using Smart-ID
		\item{Integration with mobile sign} - ability to identify user using mobile sign
	\end{enumerate}

\section{Project Team}
	\subsection{Roles}
	\begin{enumerate}
		\item{Project manager} - develops project plan, lead and manage project team, controls a scope of the project by adding and changing requirements
		\item{Scrum master} - ensures that in software development process are used best agile practices, manages the product backlog, helps with communication
		\item{Business analyst} - responsible for ensuring that requirements are captured and documented correctly before a solution is developed.
		\item{Software architect} - collaborates with analyst to determine software requirements, creates software design and resolves issues according to it.
		\item{Senior software developer} - leads the development process, ensures that development team is providing clean and efficient code
		\item{Software developer} - develops system by writing clean and efficient code
		\item{Tester} - identifies target test items, defines and executes test cases, gathers and manages test data, ensures, that the solution meets business requirements and it is free of errors and defects.
		\item{User Experience specialist} - creates user flows, wireframes, mockups and prototypes that lead to intuitive user experiences; identify user experience problems and suggest solutions.
	\end{enumerate}
	
	\subsection{Team size}
	Team is formed by 1 project manager who is also a scrum master, 1 test engineer who also performs as business analyst, 1 senior software developer who also acts as software architect, 2 software developers and 1 user experience specialist.

\section{Budget}
This section includes all necessary costs for the project: employee salaries, equipment, licenses, third-party services, etc.

\begin{center}
	\setstretch{1.0}
	\small
	\begin{longtable}{|L{2cm}|C{1cm}|C{1cm}|C{1cm}|C{1cm}|C{1cm}|C{1cm}|L{1cm}|}
		\caption{Budget item and cost across the duration of the project}
		\label{table:EmployeeSalary}
		\\ \hline
		Budget item &
		\multicolumn{1}{c|}{Month 1} &
		\multicolumn{1}{c|}{Month 2} &
		\multicolumn{1}{c|}{Month 3} &
		\multicolumn{1}{c|}{Month 4} &
		\multicolumn{1}{c|}{Month 5} &
		\multicolumn{1}{c|}{Month 6} &												
		\multicolumn{1}{c|}{\textbf{Total, \euro}} \\ \hline
		Project Manager Salary &
		3000&
		3000&
		3000&
		3000&
		3000&
		3000&
		\textbf{18000}\\ \hline
		UX Designer Salary &
		2400&
		2400&
		2400&
		2400&
		2400&
		2400&
		\textbf{14400}\\ \hline
		Test Engineer Salary &
		2000&
		2000&
		2000&
		2000&
		2000&
		2000&
		\textbf{12000}\\ \hline
		Software Developer 1 Salary &
		2800&
		2800&
		2800&
		2800&
		2800&
		2800&
		\textbf{16800}\\ \hline
		Software Developer 2 Salary &
		2800&
		2800&
		2800&
		2800&
		2800&
		2800&
		\textbf{16800}\\ \hline		
		Senior Software Developer Salary &
		4100&
		4100&
		4100&
		4100&
		4100&
		4100&
		\textbf{24600}\\ \hline
		IntelliJ Community&
		0&
		0&
		0&
		0&
		0&
		0&
		\textbf{0}\\ \hline
		Macbook Pro&
		2400&
		-&
		-&
		-&
		-&
		-&
		\textbf{2400} \\ \hline
		Axure RP 9 Pro License &
		25&
		25&
		25&
		25&
		25&
		25&
		\textbf{150} \\ \hline
		Jira Software Standard (6 users)&
		42&
		42&
		42&
		42&
		42&
		42&
		\textbf{252} \\ \hline		
		Zephyr Plugin&
		10&
		10&
		10&
		10&
		10&
		10&
		\textbf{60} \\ \hline
		5 Lenovo ThinkPad E595 models (Windows 10)&
		5 x 700&
		-&
		-&
		-&
		-&
		-&
		\textbf{3500} \\ \hline 
		Office rent
		900&
		900&
		900&
		900&
		900&
		900&
		900&
		\textbf{3600}\\ \hline
		Office 365 E3 (6 users)
		120&
		120&
		120&
		120&
		120&
		120&
		120&
		\textbf{720}\\ \hline	
		Marketing&
		20000&
		-&
		-&
		-&
		-&
		-&
		\textbf{20000} \\ \hline			
		\multicolumn{7}{|r|}{\textbf{TOTAL:}}&
		\textbf{93282}\\ \hline
\end{longtable}
\end{center}

\section{Project profit calculations}

\section{Risks and Responses}
Risks are divided into two main categories: positive risks and negative risks. All risks have a probability and an impact to the project.

\subsection{Project specific risks}

\subsection{Market research if people are interested to the product}

\subsection{Negative Risks}
Table \ref{table:NegativeRisksReponses} covers risks with negative impact to the project and the recommended response/s to mitigate said risk.
\newpage
\begin{center}
	\setstretch{1.0}
	\small
	\begin{longtable}{|C{1cm}|L{4.55cm}|C{2cm}|C{2cm}|L{4.55cm}|}
		\caption{Negative Risks and Responses}
		\label{table:NegativeRisksReponses}
		\\ \hline
		Risk code &
		\multicolumn{1}{c|}{Risk Description} &		
		Probability &
		Impact &
		\multicolumn{1}{c|}{Response} \\ \hline
		NR1 &
		\textbf{Unfamiliarity with development process used}. Team members unfamiliar with development process will have a slower workflow resulting in longer delivery times&
		Low &
		Medium &
		\textbf{Team will perform a bi-weekly retrospective meeting} at the end of the week. These retrospectives should create a safe space for team members to share their honest feedback on what's going well and what isn't, and to generate a discussion around what could be improved next time regarding the development process during the sprint.		\\ \hline
		NR2 &
		\textbf{Inefficiency of the development process}. inefficient development process lead time and lowers deliverable quality.&
		Low&
		High&
		{\textbf{Perform bi-monthly squad health checks} to gather what each team member feels about the following aspects: delivering value, easy to release, health of codebase, mission, speed, suitable process, support and teamwork. Results will allow to make necessary plans on which aspect/s to improve resulting in higher team motivation and synergy as well as adjusting the unfamiliar.\newline \textbf{Collect data of KPIs}, such as, registered defects per day, commits per day, what tasks are assigned to which member, lead time, etc. in order to follow the current state of the development process and see if made changes affected those KPIs}\\ \hline		
		NR3 &
		\textbf{Low team motivation, synergy and/or coordination}. Low motivation, synergy and/or coordination results in slower lead time, lack of innovation and is prone to higher no. of errors during the execution of the project &
		Medium&
		Medium&
{\textbf{Perform bi-monthly squad health checks} to gather what each team member feels about the following aspects: fun, learning, pawns or players, support and teamwork. Results will allow to make necessary plans on which aspect/s to improve resulting in higher team motivation and synergy.\newline \textbf{Promote an environment to nurture T-shaped professionals} resulting in increased team synergy and coordination}\\ \hline
		NR4 &
		\textbf{Insufficient team skill/knowledge regarding tech-stack}. A Lack of skill and knowledge of the tech stack used in a project results in higher amount and severity of bugs, slower development time and bad practices&
		Medium&
		High&
		Perform a different tech analysis in order to gauge the difficulty to master said tech, it's ease of use, functionality, how comfortable is the team to use said tech and so on. Performed analysis will let to pick the tech that is not only suited, but is also familliar to the team members.
		Organise knowledge sharings/workshops at the begining of the project in order for the team to fammiliarize themselves to the tech-stack\\ \hline	
		NR5 &
		\textbf{System development stallment due to covid.} Team members contracting COVID-19 will slow down and eventually stop the development of the system.&
		Medium&
		High&
		\textbf{Follow and enforce government decisions regarding the epidemic within the team.} Adhere to health rules and regulations. Allow team members to work remotely, follow safety guidelines and enforce social distancing, wearing of masks, etc., within the team.\\ \hline 
		NR6 &
		\textbf{Failure to marked advertisements to target audience}. Due to not defining in our project scope market research, there is a high chance to completely miss our target audience	 &
		High &
		High &
		\textbf{Outsource the market research and consult regarding marketing to another company.} Consult an advertisement company regarding advertising our product as well as outsource market research to define our target audience.  \\ \hline
	\end{longtable}
\end{center}


\subsection{Positive Risks}
Table \ref{table:PositiveRisksReponses} section risks with positive impact to the project and the recommended response/s to achieve those risk.

\begin{center}
	\setstretch{1.0}
	\small
	\begin{longtable}{|C{1cm}|L{4.55cm}|C{2cm}|C{2cm}|L{4.55cm}|}
		\caption{Positive Risks and Responses}
		\label{table:PositiveRisksReponses}
		\\ \hline
		Risk code &
		\multicolumn{1}{c|}{Risk Description} &		
		Probability &
		Impact &
		\multicolumn{1}{c|}{Response} \\ \hline
		PR1 &
		\textbf{Overwhelming customer demand} Due to high customer demand our resources in production might be overwhelmed. Because of this, we will not be able to provide the necesarry resources to all our customers resulting in customer dissatisfaction. &
		Low &
		Medium &
		Develop scalable infrastructure in order to meed high customer demand.\\ \hline		
		PR2 &
		\textbf{Project is under budget}. Due to a potential flaw in the planning and budgeting process we have resulted in finances that are not assigned to any budget item.  &
		Medium &
		High &
		Budget will be reevaluated and spare finances will be invested where deemed needed.\\ \hline			
	\end{longtable}
\end{center}


\end{document}
